\documentclass[17pt, a4paper]{extarticle}

\usepackage{xcolor}

\usepackage[margin=2cm]{geometry}

\parindent=0.5\baselineskip
\parskip=0.4\baselineskip

\usepackage[sfdefault]{AlegreyaSans}
\renewcommand*\oldstylenums[1]{{\AlegreyaSansOsF #1}}


\pagecolor{darkgray}
\pagestyle{empty}

\begin{document}

\color{lightgray}

On entend souvent parler du nucléaire dans les débats contemporains, du problème que posent leurs déchets, de pourquoi il faut militer contre le nucléaire et de sa dangerosité. L'autre point de vue se fait aussi entendre. Certes il y a des déchets, mais c'est la seule méthode actuelle pour créer une quantité suffisante d'électricité sans émettre de gaz à effets de serre. J'ai souvent penché de ce côté du débat, allant jusqu'à me demander pourquoi si peu de pays passaient au nucléaire.

Cet été, j'ai visité le musée Marie Curie dans le 5e arrondissement de Paris. Ainsi, ce sujet est encore frais dans mon esprit. De plus, j'avais précédemment vu le film {\itshape Radioactive} au cinéma en 2019, un biopic de la vie de Marie Curie. Ces deux derniers éléments me donnent une base conséquente de connaissances sur l'histoire de la radioactivité, son fonctionnement, ses dangers. Les thèmes de l'énergie nucléaire et de la radioactivité sont fortement liés, et ont toujours suscité mon intérêt. C'est pourquoi j'ai décidé d'en faire le sujet de mon travail d'écriture créative, en étudiant les polémiques et en critiquant les différents avis, tout en me fondant au maximum sur des faits scientifiques et historiques.

En ce qui concerne la forme de mon projet d'écriture, cet article sera constitué d'un corps textuel agrémenté d'anecdotes, de faits historiques, de définitions et d'images permettant de faire des pauses dans la lecture afin de la rendre plus légère et agréable. Cet article pourra sensibiliser les jeunes lecteurs sur les questions et les enjeux du nucléaire. Varier les supports me semble en effet pertinent pour lutter contre les longs textes noir sur blanc, qui demandent à mon sens trop de concentration et d'effort pour les lire en entier.

\end{document}